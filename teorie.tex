\chapter{Teoretický úvod}
\label{sec:teorie}

\noindent Forma závěrečné práce je popsaná na \href{ https://kdaiz.fjfi.cvut.cz/studium/bakalarske-studium/bakalarska-prace/}{stránkách KDAIZ}.

Teoretický úvod obsahuje stav problematiky. Protože úvod má být krátký, state-of-the-art část přijde až sem.

\section{Nástroje}
\label{sec:nastroje}

\noindent Nechcete-li si nic dalšího instalovat, lze práci psát v prohlížeči pomocí webového nástroje Overleaf --- tato šablona je dostupná přes následující odkaz; stačí si ji otevřít a přes \texttt{File/Make a copy} uložit kopii projektu do Vašeho účtu:

\begin{quotation}
\href{https://www.overleaf.com/read/znrvrtjfkhbg#30e8b1}{https://www.overleaf.com/read/znrvrtjfkhbg\#30e8b1}. 
\end{quotation}

Overleaf zároveň umí spolupracovat s nejrozšířenějšími systémy pro správu citací, viz \href{https://www.overleaf.com/blog/639-tip-of-the-week-overleaf-and-reference-managers}{Tip of the Week: Overleaf and Reference Managers}. 

Pokud preferujete mít vše u sebe, šablona je též k dispozici v git repozitáři na URL:

\begin{quotation}
    \href{https://github.com/vaclavstepan/kdaiz-zp-template}{https://github.com/vaclavstepan/kdaiz-zp-template}
\end{quotation}

Skvělým úvodem do použití systému \LaTeX\ je kniha pana Satrapy \LaTeX\ pro pragmatiky \cite{satrapa_latex_2011}. V \LaTeX\ je také k dispozici spousta balíčků maker zjednodušujících práci. Například nechcete-li si lámat hlavu, jak psát správně stupně Celsia,  \href{https://texdoc.org/serve/siunitx/0}{siunitx}.

K sazbě a typografii lze doporučit web \href{http://www.liteera.cz}{Litéra} (pro češtinu) a pro práce v angličtině pak Buttericks's Practical Typography~\cite{butterick_matthew_buttericks_nodate}. 
Jazyková pravidla pro práce v češtině viz.~\cite{ijp}.

Pro doplnění tvrdých mezer za jednopísmenné předložky můžete využít buď programů \href{https://petr.olsak.net/ftp/olsak/vlna/}{vlna} od \href{https://petr.olsak.net/}{Petra Olšáka}, nebo např. v AucTeX pro Emacs pomocí makra \texttt{tildify}. Vlna je také součástí distribuce TeXLive.

Budete-li chtít psát bez připojení k síti, budete potřebovat \LaTeX\ (třeba z~\href{https://tug.org/texlive/}{TeXlive}) pro překládání zdrojových souborů do PDF a nejspíš \verb?git? pro správu verzí. K~distribuované správě verzí v~\verb?git? se více dozvíte třeba v knize Scotta Chacona a Bena Strauba~\cite{chacon_pro_2014}.

\subsection{Příprava obrázků}
\label{sec:obrazky}

Pro zpracování obrázků a ilustrací by se vám mohly hodit některé z následujících volně dostupných nástrojů, podle formy a obsahu:
\begin{description}
\item[Vektorové ilustrace] --- \href{https://inkscape.org/cs/}{Inkscape}
\item[Bitmapové obrázky] --- \href{https://gimp.org/}{GIMP}
\item[Diagramy a schémata] --- \href{https://www.yworks.com/products/yed}{yEd}
\end{description}

